JTeX扩展了的catcode码:16=jletter。
pTeX扩展的catcode码:16=kanji,17=kana,18=other_kchar。
upTeX扩展的catcode码:16=kanji,17=kana,18=other_kchar,19=hangul。
PUTeX并没有扩展catcode码。

JTeX使用subfont机制,内建。
pTeX/upTeX/PUTeX使用parallel-font机制。
XeTeX/LuaTeX使用了native-font机制。

在使用parallel-font机制的引擎中:
pTeX/upTeX使用的字体机制没有match机制,需要手动;
PUTeX使用的字体有match-table机制。

XeTeX/LuaTeX的native-font机制不利于字体fallback。
LuaTeX-ja采取了range形式来模拟parallel-font机制。

Omega/Aleph不采用native-font,引入了OFM,DBCS。

汉字字体(CID)需要调整字体的基线位置。
在PDF中,字体可以设定横向和竖向的属性。
在PDF中,实现其他字体方向需要设定转置矩阵。
在PDF中,CID字体的参数需要写入DW等数组中。
在TeX中,需要自动进行标点压缩,使用FreeType。
在TeX中,需要计算字体的weight。
在TeX中,需要进行设定字体的渲染模式。
在TeX中,需要实现割注(warichu)的算法。
在TeX中,需要实现圈点的算法。
\bye