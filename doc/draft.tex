1. 关于图像的东西。AMS的ftp://ftp.ams.org/ams/author-info/documentation/creating-graphics.pdf
2. 关于Arabic的,生成的PDF中需要能够复制出来,tagged PDF,参考PDFlib中的示例
3. PDFlib公司出版的Die PostScript und PDF-Bibel需要看一下,内容不错
4. LuaTeX的历史:MKII-MkIV, The History of LuaTeX
5. tri-directional typesetting,NTS文档中提到的一个事情
6. ELECTRONIC PUBLISHING, VOL. 2(3), 3-26, triroff, an adaptation of the device-independent troff for formatting tri-directional text
   http://citeseerx.ist.psu.edu/viewdoc/download?doi=10.1.1.27.4329&rep=rep1&type=pdf
7. Comments on the Future of TeX and METAFONT
   http://www.tug.org/TUGboat/tb11-4/tb30beebe-future.pdf
8. Alexander Berdnikov. Russian Typographical Traditions in Mathematical Literature
9. Ulrik Vieth. Math typesetting in TeX: The good, the bad, the ugly
10. 专利号:200610112738。一种文字排版的方法
    专利号:200710121797。一种图文的自动排版方法
    专利号:201010242739。电子文档排版方法及装置
    专利号:201010548068。排版方法和装置
    专利号:200810240307。一种排版文件的处理方法及装置
    专利号:200910090195。一种文书文件的排版方法及装置
    专利号:200810239732。绘制符号花边的方法和装置
11. United States Patent: Optimal Line Break Determination.
12. Remco R. Bouckaert. A Probabilistic Line Breaking Algorithm.
    www.cs.waikato.ac.nz/~remco/line.pdf
13. liblinebreak, vimgadgets.sourceforge.net/liblinebreak/
14. 韩世勇,Microtypographic extensions to the TeX typesetting system
    www.pragma-ade.com/pdftex/thesis.pdf
15. 韩世勇,Margin Kerning and Font Expansion with pdfTeX
16. 韩世勇,A closer look at TrueType fonts and pdfTeX
17. Computer-Aided Design. Vol. 27, No. 3, pp193-207, 1996. Oriental Character Font Design by a Structured Composition of Stroke Elements.
18. Tung Yun Mei. LCCD, A Language for Chinese Character Design.
19. John Hobby, Gu Guoan. A Chinese METAFONT.
20. ELECTRONIC PUBLISHING, VOL. 6(2), 67-91 (JUNE 1993). Structure extraction and automatic hinting of Chinese outline characters
21. John Plaice on Omega and Beyond (an interview)
22. Omega and OpenType Fonts
23. Adobe Font Metrics File Format Specification
24. Multiple Master math extension fonts
25. Breaking Paragraphs into Lines
26. The Oxford Guide to Style
27. Type Rules
28. Unicode Explained
29. Unix Text Processing
30. Words into Type
31. Practical Printing
32. PostScript Language Reference
33. Literate Programming
34. Adobe CJKV Character Collections and CMaps for CID-Keyed Fonts
35. Building CMap Files for CID-Keyed Fonts
36. CID-Keyed Font Technology Overview
37. SING Glyphlet Production: Tips, Tricks & Techniques
38. OpenType-CID/CFF CJK Fonts: ‘name’ Table Tutorial
39. Designing Multiple Master Typefaces
40. Font Naming Issues
41. CID-Keyed sfnt Font File Format for the Macintosh
42. CJKV Information Processing
43. Computer Systems A Programmer's Perspective, 2E (CSAPP2e)
44. LGPL Hyphenation Library
    http://www.linux-magazine.com/Online/News/LGPL-Hyphenation-Library
    http://swolter.sdf1.org/software/libhyphenate.html
45. libhyphenation.a in Android.
46. Libhnj, a library for high quality hyphenation and justification
47. TeXHyphenator-J
48. Using TeX hyphenation patterns in OpenOffice.org
    http://wiki.openoffice.org/wiki/Documentation/SL/Using_TeX_hyphenation_patterns_in_OpenOffice.org
49. Knuth linebreaking elements for Formatting Objects
    http://www.leverkruid.eu/GKPLinebreaking/elements.html
50. Unicode line breaking rules: explanations and criticism
    http://www.cs.tut.fi/~jkorpela/unicode/linebr.html
51. GLOBAL MULTIPLE OBJECTIVE LINE BREAKING, ALEX HOLKNER
    http://bowman.infotech.monash.edu.au/~pmoulder/line-breaking/holkner-multiobjective-linebreaking-paper.pdf
52. dvipdfmx 文件解说:
    agl.c | Adobe Glyph List
    bmpimage.c | BMP文件支持,写入PDF的object,校验函数,colorspace
    cff.c | CFF格式字体处理。
    cff_dict.c | CFF字体词典。有启发性的地方:UnderlinePosition,UnderlineThickness
    cff_stdstr.h | Glyph的名字。
    cff_types.h | 定义了读取CFF字体需要用到的类型定义。
    cid.c | 对于CIDType0和CIDType2的字体处理。六种CID类型。
          | 注意各种PDF版本支持的ROS。
          | Render Mode是独立于字体的。
    cid_basefont.h | 这是所谓的基本CID字体系列,但是不是真正意义上的,是基于Adobe阅读器的基本CID字体。
                   | 注意:各个版本的Adobe Reader使用CID字体名并不是一样的。
                   | 非嵌入式的CID字体的作用是减少PDF体积,一般在打印机中存在。
                   | 注意/Panose的生成。
    cidtype0.c | 这个呢,主要是加入水平和垂直方向的Metrics。142行有亮点存在。
               | 注意写入ToUnicode的部分。
               | get_font_attr这个函数中也进行了一些参数的设定,尤其是StemV。
    cidtype2.c | 这个是写入TrueType的。编码比较多了。FreeType是可以进行分析的,但是子集化必须自己进行处理。
               | 单独写成库的话,一定要把子集化字体做出来,这个有利于制作字体。
    cmap.c | 这个是读取CMap的库,CMap对于精确处理编码是很有帮助的。
    cmap_read.c | 这个是一个简单的parser。
    cmap_write.c | 这是一个写入的东西,写入到PDF才能够复制出有编码的文本出来。
    cs_type2.c | 这个有关于PostScript了。
    dpxconf.c | 这个是关于尺寸的,有个叫做libpaper的库。其实没有必要用libpaper。
    dpxcrypt.c | 这个是关于加密算法的,有一个简单的MD5算法。
    dpxfile.c | 利用Kpathsea查找各种文件。
    dpxutil.c | 这里是一堆辅助函数。
    dvi.c | 这个是DVI转换成PDF的主力了。
          | 实际上,从不同的驱动来看,字体的技术已经分层了:1. 实体,2.虚拟,3,次字体。
          | 虚拟字体有助于解决多语种的问题,一个所谓unifont机制。
    dvicodes.h | 已经定义了的opcode。
    dvipdfmx.c | 主程序。
    epdf.c | 各种裁剪框。但是请注意,这是将PDF作为插图插入的情况。
    error.c | 各种报错程序。
    fontmap.c | 字体映射文件,映射是由离到合的过程。支持pdfTeX那种映射的情况。
              | pdf_load_native_font,这个函数里面有问题。
    jpegimage.c | 这个是关于JPEG文件处理的,需要查看一定的材料。
    mem.c | 关于内存管理的。
    mpost.c | 注意57行开始的注释。这是处理PostScript代码的。
    numbers.c | 操作数。
    pdfcolor.c | 关于PDF中的各种色彩。
    pdfdev.c | 关于各种PDF生成计算。注意对于未知的计算:图像模式,文本模式,字符串模式。
             | 在数据结构中就需要对TFM和实际字体进行适当的映射,而LuaTeX做的并不太好。
             | LuaTeX对于东方语言的支持在底层上面就是缺乏的,所以当然不会出现什么好结果。
    pdfdoc.c | 这里就是处理一些special代码的地方了。
    pdfdraw.c | 这个就是全局处理绘图的地方了。
    









 
